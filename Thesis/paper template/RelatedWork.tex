In this section, we review two types of related literature - digital humanities projects which build gazetteers from text and the process of identification of influential people from data.

\subsection{Gazetteers for Digital Humanities Projects}
Newspaper archives have been studied extensively for the design of search and retrieval algorithms (\cite{Shahaf_11, Gabrilovich_04a, Alonso_10, Khurdiya_11}), summarization (\cite{McKeown95, Otterbacher06, Radev97,Radev01,Radev05 }), sentiment analysis, topic modeling (\cite{Masand_92, Nallapati_04a, Radev99c}) and visualization.
Historical newspaper archives from Chronicling America\footnote{http://chroniclingamerica.loc.gov/} have been used for topic modeling during historically significant time periods\cite{yang2011topic}. Newman et. al\cite{newman2006analyzing} use a combination of Statistical Topic Modeling and Named Entity Recognition for analyzing entities and topics from a news articles dataset. They also create networks based on the relationships among the entities.
Llyod et. al \cite{lloyd2005lydia} discuss their approach for designing a news analysis system\footnote{http://www.textmap.com} where information about several types of entities can be searched. They perform temporal and spatial analysis and present time series popularity graphs based on the number of references and co-reference names for the entity. %Above mentioned research works stress on person entities in a newspaper environment but do not focus on finding influential entities in which respect our research is different from their work.

Several digital humanities projects that have used machine learning and natural language processing techniques to learn from historic newspaper archives are relevant to this work -- the libraries of Richmond and Tufts have examined the Richmond Times Dispatch during the civil war years for more than two decades and their work focuses on automatic identification and analysis of full OCR text in newspapers to provide advanced searching, browsing and visualization\cite{crane2006challenge}. The focus of this work was on named entity extraction and ten categories prominent in these newspapers were studied including ship names, railroads, streets and organizations. In an earlier project at the universities, the Perseus project \cite{smith2002detectinga, smith2002detectingb, smith2001disambiguating}, a general system to extract dates and names from text was developed in order to detect significant events in document collections. 

Developing gazetteers from news articles is a well established technique - different types of gazetteers are discussed under the General Architecture for Text Engineering (GATE\footnote{http://gate.ac.uk/sale/tao/splitch13.html}) framework. It defines a gazetteer as a set of lists containing names of entities (such as cities, organizations, days of the week, etc) which are used to find occurrences of these names in text. We use this definition to develop our People Gazetteer that finds person name entities from a news article repository and associates each unique person entity with the list of articles in which they occur.

Gazetteer lists are also discussed in \cite{carlson2009learning} where they are used for learning name entity tagger using partial perceptron and aid in performing better NER compared to CRF based entity taggers. \cite{zhang2009novel} discuss automatic generation of gazetteer list by finding entities with similar type labels from Wikipedia articles. The evaluation is done over scientific domain of Archeology considering subject, temporal terms and location as named entities but no evaluation is presented for person entities. \cite{allen2013toward} describe an exploratory study for developing an interactive directory for the town of Norfolk, Nebraska for the years 1899 and 1900. It focuses on providing structured and richer information about the person entities by linking their occurrences with associated events described in historical newspapers. Their entity-based directory is similar to our people gazetteer although we do not restrict our problem to any specific town or significant era nor do we consider any specific town directory to begin with. 
%They do not provide any implementation details or results and do not talk about procedure of finding influential persons from their interactive directory. 
%There is also no relevant work that builds or uses historical person names gazetteer list for data mining that we know of.

\subsection{Influential People Detection}
In prosopography, identification of the ``social elite" plays an important role. Their experience and personal testimonies may be reported at length in newspaper articles. 

In the context of machine learning and data mining, influential people detection has been mostly done in the field of social networks, marketing and diffusion research.
Kempe et. al \cite{kempe2003maximizing} present work on choosing the most influential set of nodes  in a social network in order to maximize user influence in the network. They consider spread of influence from an influential node cascading through a network which further influences other neighborhood nodes. In this research, we do not focus on the network formed by person entities. Lerman et. al \cite{lerman2010using} define popularity of a news story in terms of number of reader votes received by it. Popularity over time is based on voting history and the probability that a user in a list will vote. To identify influential bloggers, \cite{agarwal2008identifying} quantify influence of each blogger by taking the maximum of the influence scores of each blog posted by the blogger. The influence score is calculated using the number of posts that refer to the blog, number of comments on the blog, number of other posts that the blog refers to and length of the blog. Influential blogger categories are also created based on the temporal patterns of blog posting. 

\cite{cha2010measuring} describe another set of measures for detection of top influential users on Twitter using number of retweets, mentions and followers for an individual. They perform ranking based on each measure separately and use Spearman's rank correlation coefficient to find correlation among ranks and effect of each measure contributing to a person's influence. The influence ranks of topmost influential users on Twitter are presented across various topics as well as time.

In all of the above, the goal is to measure influence or popularity -- however, these cannot be directly adapted to the gazetteer or newspaper articles. 
