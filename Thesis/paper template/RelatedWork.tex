
Different types of gazetteers are discussed in \footnote{http://gate.ac.uk/sale/tao/splitch13.html}. They define gazetteers as set of lists containing names of entities such as cities, organizations, days of the week, etc. along with their types. They use gazetteer either as set of entity list or as a processing resource which is used to find occurrences of the entity names in text, e.g. for the task of named entity recognition. We use this definition to develop our People Gazetteer as a processing resource that finds person name entities from the news articles repository, associates each unique person entity from news articles with a list of articles of its occurrence and their respective topic.

Gazetteer lists are also discussed in \cite{carlson2009learning} where they are used for learning name entity tagger using partial perceptron and aid in performing better NER compared to CRF based entity taggers.
\cite{zhang2009novel} discuss automatic generation of gazetteer list by finding entities with similar type labels from Wikipedia articles which can further be used for the purpose of NER. The evaluation is done over scientific domain of Archeology considering subject, temporal terms and location as named entities but no evaluation is presented for person entities. \cite{allen2013toward} describe an exploratory study of developing an interactive directory for the town of Norfolk, Nebraska for the years 1899 and 1900 that focuses on providing structured and richer information about the person entities that occur in the town directory by linking their mention and events associated with them in the historical newspapers. Their entity-based directory is similar to our people gazetteer although we do not restrict our problem to any specific town or significant era nor do we consider any specific town directory to begin with. They do not provide any implementation details or results and do not talk about procedure of finding influential persons from their interactive directory. 
%There is also no relevant work that builds or uses historical person names gazetteer list for data mining that we know of.

Several digital humanities projects that have used machine learning and natural language processing techniques to learn from historic newspaper archives are relevant to this work -- the libraries of Richmond and Tufts have examined the Richmond Times Dispatch during the civil war years for more than two decades and their work focuses on automatic identification and analysis of full OCR text in newspapers to provide advanced searching, browsing and visualization\cite{crane2006challenge}. The focus of this work was on named entity extraction and ten categories prominent in these newspapers were studied including ship names, railroads, streets and organizations. In an earlier project at the universities, the Perseus project \cite{smith2002detectinga, smith2002detectingb, smith2001disambiguating}, a general system to extract dates and names from text was developed in order to detect significant events in document collections. 

Historical newspaper archivesfrom Chronicling America have also been used for topic modeling in order to learn topics of interest during historically significant time periods\cite{yang2011topic}. \cite{newman2006analyzing} use a combination of Statistical Topic Modeling and Named Entity Recognition techniques for analyzing the entities, topics trends and topics that relate entities mentioned in a news articles dataset. They also create networks based on the topic model based relationships among the entities.
\cite{lloyd2005lydia} discuss their approach for designing a news analysis system \footnote{http://www.textmap.com} where information about several types of entities can be searched. They allow searching over all entities found in the news sources, present juxtaposition for each entity, i.e., other entities mentioned in context, temporal and spatial analysis, popularity time series graph in terms of number of number of references and co-reference names for the entity.
 Above mentioned research works stress on person entities in a newspaper environment but do not focus on finding influential entities in which respect our research is different from their work.

Influential people detection has been mostly done in the field of social networks, marketing and diffusion research.
\cite{kempe2003maximizing} work on choosing the most influential set of nodes  in a social network in order to maximize user influence in the network. They consider spread of influence from an influential node cascading through a network which further influences other neighborhood nodes but we do not consider the case of a network of connected person entities in our research where influence score of a person entity could be influenced by that of its neighboring person entity nodes. We consider each person entity connected with a list of articles of its occurrence instead and measure the person entity's influence score by finding the effect of influence of each article in that list.
 

\cite{lerman2010using} define popularity of a news story in terms of number of reader votes received by it and predict popularity of a news story over time based on voting history and the probability that a user seeing a story at specific position in a list will vote on it. 
A more relevant work regarding detection of influential people is presented in \cite{agarwal2008identifying} where influential bloggers are identified on a blog site. Influence of each blogger is quantified by taking maximum of the influence scores of each blog posted by a blogger. The influence score of each blog is calculated using parameters of importance in a blogsite like number of posts that refer to the blog, number of comments on the blog, number of other posts that the blog refers to and length of the blog. Influential blogger categories are also created based on the temporal patterns of blog posting by bloggers. 

\cite{cha2010measuring} describe another set of measures for detection of top influential users on Twitter using number of retweets, mentions and followers for an individual. They perform ranking based on each measure separately and use Spearman's rank correlation coefficient to find correlation among ranks and effect of each measure contributing to a person's influence. The influence ranks of topmost influential users on Twitter are presented across various topics as well as time.

In the above mentioned works, although the problem description matches with our research problem but the parameters defined to measure influence or popularity cannot be directly used in the newspaper environment. 
