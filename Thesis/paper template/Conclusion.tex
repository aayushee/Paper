The problem of finding influential people from a historical OCR news repository has been studied to aid quantitative prosopography research. The solution framework comprises of development of a people gazetteer for facilitating the process of influential person detection. A novel algorithm for detecting influence has been presented which examines spelling correction of noisy OCR, person name entity recognition, topic detection and heuristics to measure influence and rank of people mentioned in the articles. Our algorithm has been tested on approximately 40000 people discussed in historic newspapers. The Tsar of Russia (Alexander III), the first and fourth Prime Ministers of Canada (John McDonald and John Thompson), and English soldiers serving in World War I (Captain Hankey) are among the influential people identified by the algorithm.



\section{Acknowledgement}
This work was initially supported by the National Endowment of Humanities grant no. NEH HD-51153- 10. The authors would like to thank Barbara Taranto and Ben Vershbow from the NYPL Labs for providing the article level newspaper data and Manoj Pooleery, Deepak Sankargouda and Megha Gupta for setting up the database used in this research.

% In studying this novel problem, our main aim was to develop a complete solution framework for this problem and present insights from the results obtained.
% We made novel contributions to the problem solution by developing a people gazetteer for facilitating the process of influential people detection and finally defining parameters and measures in the newspaper community to obtain the ranked list of influential people.
%Spelling correction algorithms with improved accuracy can certainly improve the influential persons results.
%Topic detection algorithms also need to be designed to enable them to deal with noisy OCR text in a better manner as some of the topics we obtained using LDA came out to be garbled and were difficult to understand in order to perform human-assigned manual labeling on them and use them further for finding similarity across articles.
%We didn't consider Named Entity Disambiguation into account while developing the people gazetteer for detection of influential people which is a difficult problem in itself since it is hard to disambiguate among persons with similar names that can occur in multiple topic related articles in newspapers. The problem presented in this paper requires research into better spelling correction, named entity recognition, topic detection algorithms and stricter measures of calculation of influence score and ranking of influential persons.
%
% The parameters we defined for measuring influence scores of persons in news articles are based on heuristics and can be re-weighted according to user requirements or new parameters can be defined based on the characteristics of an OCR newspaper dataset making it an open research problem.
%
%Non-heuristic based estimation for finding influential persons can also be done using optimization approaches such as unsupervised multiple instance clustering\cite{zhang2009m3ic}\cite{zhang2009multi} but they need to be adapted in order to be used in a large scale environment.