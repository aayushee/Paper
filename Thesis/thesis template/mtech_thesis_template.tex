\documentclass[letterpaper,11pt]{report}
\usepackage[boxed]{algorithm2e}
\usepackage{algpseudocode}
\usepackage{fullpage}
\usepackage{verbatim}
\usepackage{cite}
\usepackage{setspace}
\usepackage{fancyhdr}
\usepackage{amsmath}
 \usepackage{booktabs}
\usepackage{subfigure}


%\usepackage{noReferences}
\usepackage[small]{caption}
\usepackage[leftcaption]{sidecap}
\usepackage{graphics}
\usepackage{color}

\usepackage{hyperref}

%\usepackage{natbib}

\usepackage[dvips]{graphicx}
 % define the title
\graphicspath{ {images/}}

% Paper conservation layout. Long live the trees!!
\setlength{\oddsidemargin}{-0.4mm} % 25 mm left margin
\setlength{\evensidemargin}{\oddsidemargin}
\setlength{\textwidth}{160mm}      % 25 mm right margin
\setlength{\topmargin}{-5.4mm}     % 20 mm top margin
\setlength{\headheight}{5mm}
\setlength{\headsep}{5mm}
\setlength{\footskip}{10mm}
\setlength{\textheight}{237mm}     % 20 mm bottom margin


\setlength{\parskip}{1ex}
\parindent 0in
\def\SAONE{Specific Aim 1}
\def\SATWO{Specific Aim 2}
\def\SATHREE{Specific Aim 3}
\def\SAFOUR{Specific Aim 4}

\def\title{Mytitle}
%\def\titletwo{Thesis Proposal Title Line 2}

\begin{document}




\def\addrone{Your address}
\def\addrtwo{Your city}

\def\degree{M.Tech. in Computer Science with Specialization in Data Engineering}


\def\submissiondate{April 01, 2014}

\def\supervisorone{Haimonti Dutta}

\def\supervisortwo{XXXX}

\def\supervisorthree{YYYY}


%\def\supervisorfour{XXX XXXX }


%\def\supervisorfive{YYY YYYY}

\thispagestyle{empty}

\begin{center}

{\LARGE \bf {Mining Influential People from a Historical News Repository }

 }  
 \vspace{.3in}
 
 {\Large{Student Name: Aayushee Gupta}} \\  
 \vspace{.1in} 
 IIIT-D-MTech-CS-DE-12-030 \\

% Nov 30, 2011 \\
  
    \vspace{.35in}

  \vspace{.25in}

{Indraprastha Institute of Information Technology\\
New Delhi}

\vspace{.35in}  {\underline{Thesis Committee} \\ \supervisorone         
   \\ \supervisortwo \\ \supervisorthree }\\ \vspace{.35in}


 {Submitted in partial fulfillment of the requirements \\for the Degree of M.Tech. in Computer Science, \\ with specialization in Data Engineering}

\vspace{.2in}

\copyright 2014 SSSSSS SSSSSS \\ All rights reserved \\
\vspace{.8in}


\end{center}


\newpage

\pagestyle{empty}
\vspace*{7.1in} 
Keywords: Gazetteer, Text Mining, Information Retrieval, OCR, Spelling Correction, Historical data, Influential people detection

\newpage

\begin{center}
\section*{Certificate}\label{section:certificate}
\end{center}
%\vspace{3in}
This is to certify that the thesis titled \textbf{``Finding Influential People from Historical new repository"} submitted by \textbf{Aayushee Gupta} for the partial fulfillment of the requirements for the degree of \emph{Master of Technology} in \emph{Computer Science \& Engineering} is a record of the bonafide work carried out by her under our guidance and supervision in the Data Engineering group at Indraprastha Institute of Information Technology, Delhi. This work has not been submitted anywhere else for the reward of any other degree. \\ \vspace{0.5in}

\textbf{Dr. Haimonti Dutta}\\
\textbf{Dr. Srikanta Bedathur}\\
\textbf{Dr. Lipika Dey}\\
\textbf{Indraprastha Institute of Information Technology, New Delhi}
%\doublespacing

\begin{abstract}

 Historical newspaper archives provide a wealth of information. They
are of particular interest to genealogists, historians and scholars
for People Search.

 In this thesis, we design a People Gazetteer from
the noisy OCR text of historical newspapers and identify “influential”
people from it. A People Gazetteer is a dictionary of personal names;
each entry of the gazetteer is  a tuple containing a person name and a
list of articles in which his name occurs.

To build the People Gazetteer, we first spell correct the noisy text
using an edit distance based algorithm. A novel N-gram based
evaluation algorithm is designed for measuring the performance of the
spell corrector. Next, a Named Entity Recognizer is run on the text of
each article to identify person entities and an LDA-based topic
detector to assign categories to articles. To identify influential
people from People Gazetteer across each category, we define the notion
of an Influential Person Index (IPI) and rank based on it. Our corpus
is a sample of 14020 newspaper articles (roughly two months’ data)
obtained from “The Sun” newspaper in the Chronicling America project.

\end{abstract}

\newpage
\pagestyle{empty}


\newpage



\section*{Acknowledgments}\label{section:acknowledgments}
\pagestyle{plain}
\pagenumbering{roman}

XXXXXX XXXXXX XXXXXX XXXXXX XXXXXX XXXXXX XXXXXX XXXXXX XXXXXX XXXXXX XXXXXX XXXXXX XXXXXX XXXXXX XXXXXX XXXXXX XXXXXX XXXXXX XXXXXX XXXXXX XXXXXX XXXXXX XXXXXX XXXXXX XXXXXX XXXXXX XXXXXX XXXXXX XXXXXX XXXXXX XXXXXX XXXXXX 

\newpage

\tableofcontents
\listoffigures 
\listoftables

\newpage

\newpage

\newpage
\mbox{}


%\doublespacing

\chapter{Introduction}\label{chapter:introduction}
%\pagestyle{fancy}
\pagenumbering{arabic}
\setcounter{page}{1}
\onehalfspacing



\section {Research Motivation}

Historical newspaper archives are of extreme importance for genealogists, scholars and historians. 
There have been extensive studies on text mining from newspapers that deal with clustering newspaper articles into categories\cite{dutta2011learning}, linking similar news stories related to an event \cite{khurdiya2011multi}\cite{shahaf2010connecting}, event extraction from news stories and news story summarization \cite{mckeown2002tracking}.


Historical newspapers can also be used for People Search, for example, to gain information about important people and track the complete timeline of news articles related to them rather than searching for the articles directly. News articles related to an influential person can also be linked to his/her Wikipedia entry so that a user can directly read how, when and where the person had been involved. In fact missing entries can be added to Wikipedia for such influential people across multiple topic categories.
Influential people from historical newspapers can be further studied to find patterns among them and create influential person networks so as to detect themes and learn about entities involved in historical events. This can be done by learning from temporal trends and network structure of connected influential persons and the respective articles they occur in. 


Finding influential persons from historical newspapers, thus,  opens a wide range of possibilities. But the concept of finding influential persons in a newspaper setting is a novel one and has not yet been studied extensively which emphasizes the importance of this research. Using an open source historical news OCR dataset in spite of it being extremely noisy is also another factor which has motivated us to use such a dataset for this research.


\section{Problem Description}
\label{problem}
%%DOUBT: DIFFERENCE BETWEEN AIM AND PROBLEM DESCRIPTION? DO THEY NEED TO BE MENTIONED SEPARATELY?


The goal of this research is to find and rank influential people across multiple topic categories in historical newspaper OCR archives.


%DOUBT:  IS IT OK TO MENTION THESE DETAILS HERE OR THEY SHOULD BE WRITTEN IN RELATED WORK?
The problem of finding influential people in this scenario is a novel one as much of the research work deals with identification of influential nodes in social networks or marketing and diffusion research. This research does not involve finding influential people in terms of their influence on their peers\cite{watts2007influentials} or influence propagation in a social network \cite{kempe2003maximizing} which is how the concept of influence is used in general.   

 An influential person can be defined as ``a person whose actions and opinions strongly influence a course of events". This allows us to link an influential person with a list of articles that he/she occurs in.
 A person might be considered influential  in the newspaper environment if the person gets talked about frequently in news articles. The problem can be also be phrased as identifying and ranking popular people across various categories in the news domain. 
Popularity can be defined in other domains by counting number of votes, tweets, citations, etc. \cite{}but similar measures are not applicable in a newspaper setting where only newspaper articles mentioning multiple people are available.  
 
We divide the the problem of finding influential people into the following subproblems:
\begin{description}
\item [$\bullet $Problem 1] \label{problem:1}: Dealing with a huge dataset consisting of a large number of OCR errors.
\item [$\bullet $Problem 2] \label{2}: Develop an organized structure in order to ease the process of identification of influential people.
\item [$\bullet $Problem 3] \label{3}: Define the criteria for identifying and ranking persons as `influential" in a newspaper environment.
\end{description}
  
Each of the above problems require consideration of the dataset size and characteristics along with the newspaper environment in mind. 



%DOUBT: EITHER WRITE AS WE AIM TO ANSWER THESE QUESTIONS IN THIS RESEARCH OR WE DIVIDE THE PROBLEM INTO THESE SUBPROBLEMS...
%We aim to answer the following questions related to the problem of finding influential people with this research:
%Question 1 : How to deal with OCR data consisting of extremely noisy text for such a task?
%Question 2 : How to develop and use an organized structure for easy identification of influential people?
%Question 3 : Who are `influential persons" in a newspaper scenario and how to rank them?


%DOUBT: NOT SURE WHETHER RESEARCH FRAMEWORK SHOULD COME FIRST OR NOVEL CONTRIBUTION?
\section{Novel Contribution}
We make the following novel contributions for addressing the research problem mentioned in Section ~\ref{problem}:
\begin{enumerate}
\item Develop an evaluation algorithm for measuring the performance of spelling correction preprocessing applied to the dataset consisting of OCR errors.
\item Develop a People Gazetteer; an organized  dictionary of person names and a list of articles in which his name occurs along with the corresponding topic of each article to facilitate identification of influential people.
\item Define an Influential Person Index (IPI) and metrics for its calculation in order to identify and rank influential people.
\end{enumerate}


\section {Research Framework }

We propose the solution framework in Figure 1.1 for the purpose of finding influential people from a historical news repository. Each component of the solution framework is briefly described as follows:

\begin{enumerate}
\item \textbf {Data Gathering}:  
This is the first component of this research and describes the source of gathering data along with data characteristics and statistics. It mentions conversion of page level newspaper images into text through OCR followed by article level segmentation and description of various types of errors in the data. This component is further discussed in Chapter ~\ref{chapter:data description}.

\item \textbf {Data Preprocessing}:
This component describes the preprocessing applied on the news articles dataset. It keys out the process of spelling correction of data in detail with spelling correction algorithm, a novel algorithm for its evaluation and results. This component is further discussed in detail in Chapter ~\ref{chapter:data preprocessing}.

\item \textbf {Development of People Gazetteer}:
This component describes the process of development of people gazetteer which involves Named Entity Recognition in order to find person entities, Topic Detection using LDA to assign topics across the news articles and linking of both to obtain an organized structure. This component is discussed in detail in Chapter ~\ref{chapter:people gazetteer}.

\item \textbf {Influential Person Identification}:
This component defines an ``Influential Person Index" (IPI) that incorporates several criteria for identifying and ranking of ``influential people" across newspaper topics. Details about IPI, ranking and final results with some case studies are discussed in Chapter ~\ref{chapter:influential people detection}.

\end{enumerate}
 
\begin{figure}[h]
\includegraphics{framework3}
\caption{Research Framework diagram showing components of proposed solution}
\end{figure}  

\chapter{Related Work}


\section{Learning from newspapers}

Crowdsourcing has been used extensively in historical newspaper archives in recent years to 
digitize, create, clean and process content and provide editorial or processing interventions. For example, the Australian Newspapers Digitization Program \cite{ADNP} allows communities to explore their rich newspaper heritage by enabling free online public access to over 830,000 newspaper pages containing 8.4 million articles. The public enhanced the data by correcting over 7 million lines of text and adding 200,000 tags and 4600 comments \cite{holley_09,holley_09a,Holley_10,Holley10a}. The California Digital Newspaper Collection (CDNC)\footnote{http://cdnc.ucr.edu/cgi-bin/cdnc}, which contains 61,412 issues comprising 545,955 pages and 6,364,529 articles from newspapers published in California between 1846-1922 has also crowdsourced text correction. The National Library of Finland embraced the the idea of making crowdsourced text correction a game -- users corrected their digitized newspapers by playing the game ``Hunt the Mole!" \cite{chrons_11}. The program consists of two games featuring adventures of a mole. In Mole Hunt the players are shown two different words and they must determine as quickly as possible if the words are the same. In Mole Bridge players try to write the word that appears in the screen correctly. 

Several digital humanities projects that have used machine learning and natural language processing techniques to learn from historic newspaper archives are relevant to this work -- the libraries of Richmond and Tufts have examined the Richmond Times Dispatch during the civil war years for more than two decades and their work focuses on automatic identification and analysis of full OCR text in newspapers to provide advanced searching, browsing and visualization\cite{crane2006challenge}. The focus of this work was on named entity extraction and ten categories prominent in these newspapers were studied including ship names, railroads, streets and organizations. In an earlier project at the universities, the Perseus project \cite{smith2002detectinga, smith2002detectingb, smith2001disambiguating}, a general system to extract dates and names from text was developed in order to detect significant events in document collections. 

\cite{newman2006analyzing} use a combination of Statistical Topic Modeling and Named Entity Recognition techniques for analyzing the entities, topics trends and topics that relate entities mentioned in a news articles dataset. They also create networks based on the topic model based relationships among the entities.
\cite{lloyd2005lydia} discuss their approach for designing a news analysis system \footnote{http://www.textmap.com} where information about several types of entities can be searched. They allow searching over all entities found in the news sources, present juxtaposition for each entity, i.e., other entities mentioned in context, temporal and spatial analysis, popularity time series graph in terms of number of number of references and coreference names for the entity.
Both these research works stress on person entities in a newspaper environment but do not focus on finding influential entities in which respect our research is different from their work.


\section{Developing Gazetteers}
Different types of gazetteers are discussed in \footnote{http://gate.ac.uk/sale/tao/splitch13.html}. They define gazetteers as set of lists containing names of entities such as cities, organizations, days of the week, etc. along with their types. They use gazetteer either as set of entity list or as a processing resource used to find occurrences of the entity names in text, e.g. for the task of named entity recognition. We use this definition to develop our People Gazetteer as a processing resource that finds person name entities from the news articles repository, associates each unique person entity from news articles with a list of articles of its occurrence and their respective topic.

Gazetteer lists are also discussed in \cite{carlson2009learning} where they are used for learning name entity tagger using partial perceptron and aid in performing better NER compared to CRF based entity taggers.
\cite{zhang2009novel} discuss automatic generation of gazetteer list by finding entities with similar type labels from Wikipedia articles which can further be used for the purpose of NER. The evaluation is done over scientific domain of Archeology considering subject, temporal terms and location as named entities but no evaluation is presented for person entities.
There is also no relevant work that builds or uses historical person names gazetteer list for data mining that we know of.



We discuss more related work regarding data preprocessing using Spelling correction algorithms in Section ~\ref{spell:rw} and finding influential people in Section ~\ref{influential:rw}.






\chapter{Data Description}
\label{chapter:data description}

This chapter describes the dataset used for developing the People Gazetteer. Following sections provide details of data source, characteristics and some data statistics.

\section{Data Source} 

The dataset has been taken from Chronicling America.
\noindent \emph{Chronicling
America}\footnote{\texttt{http://chroniclingamerica.loc.gov/}} is an
initiative of the National Endowment for Humanities (NEH) and the
Library of Congress (LC) whose goal is to develop an online,
searchable database of historically significant newspapers between
1836 and 1922. The New York Public Library (NYPL) is part of this
initiative and has scanned 200,000 newspaper pages published between
1890 and 1920 from microfilm.

In order to make a newspaper available for searching on the Internet,
the following processes used in \cite{dutta2011learning} must take place: (1) the microfilm copy or
paper original is scanned; (2) master and Web image files are
generated; (3) metadata is assigned for each page to improve the
search capability of the newspaper; (4) OCR software is run over high
resolution images to create searchable full text and (5) OCR text,
images, and metadata are imported into a digital library software
program. The scanned newspaper holdings of the NYPL offers a wealth of
data and opinion for researchers and historians.

The newspaper titles and digitized pages available through the
Chronicling America website can be searched using the OpenSearch
protocol\footnote{\texttt{http://www.opensearch.org/Home}}.
Unfortunately, the current search facilities are rudimentary and
irrelevant documents are often more highly ranked than relevant ones.
The newspapers are scanned on a page-by-page basis and article level
segmentation is poor or non-existent; the OCR scanning process is far
from perfect and the documents generated from it contains a large
amount of garbled text. In a bid to serve its patrons better, the New
York Public Library employed human annotators to clean headlines of
articles and text, but the process of manually reading all the old
newspapers article-by-article and cleaning them soon became very
expensive. 

\section{Data Characteristics}
An individual OCR text article has at least one or more of the following types of spelling errors:

\begin{figure}[hbt]
\includegraphics[scale=0.75]{originalimage}
\includegraphics[scale=0.80]{ocr}
\caption{Scanned Image of a Newspaper article (left) and its OCR raw text (right)}
\end{figure}

\begin{description}
 \item[$\bullet$Real word errors] include words that are spelled correctly in the OCR text but still incorrect when compared to the original newspaper article image. For example: In Figure 3.1, the word ``coil"  has been correctly spelled in the OCR text  but should have been ``and" according to the original newspaper article. 
 \item[$\bullet$Non-real word errors] include words that have been misspelled due to some insertion, deletion, substitution or transposition of characters from a word. For eg. In Figure 3.1, the word ``tnenty" in the OCR text has a substitution error (`n' should have been `w') which is actually ``twenty" according to the original newspaper article.
 \item[$\bullet$Non-word errors] include words that have been spelled incorrectly and are a combination of alphabets and numerical characters. For example: In Figure 3.1, the word ``4anrliteii" which is a combination of alphabets and number and should have been ``confident" as per the original newspaper article.
\item[$\bullet$New Line errors] include words that are separated by hyphens where part of a word is written on one text line and remaining part in the next line. For example: In Figure 3.1, the word ``ex-ceptionally" where ``ex" occurs on one line while ``ceptionally" in the next and due to no punctuation in the text, they are treated as separate words in OCR text.
\item[$\bullet$Word Split and Join errors] include words that either get split into one of more parts or some words in a sentence get joined to a make a single word. For example: In Figure 3.1, the word ``Thernndldntesnra" in the OCR text is actually a combination of three words ``The candidates are" while the words ``v Icrory" are actually equivalent to a single word ``victory" when compared with the original news article.
\end{description} 

\section{Data Statistics}
The OCR text available from Chronicling America website is on a page by page level and no article level segmentation is provided. OCR text dataset is therefore, taken from a PostgreSQL database where article level segmentation of page-level OCR text from Chronicling America is available for two months of articles of ``The Sun" newspaper from November-December 1894 consisting of 14020 news articles with a total of 8,403,844 tokens. The newspaper database ER diagram \footnote{https://power.ldeo.columbia.edu/twiki/pub/Incubator/BodhiDBDesign/Final ERD.pdf }
is used to extract the required articles text from the database by dumping complete dataset and extracting individual articles linetext based on their unique ID. The individual text articles generated from the database do not have any punctuation and contain a large amount of garbled text containing above mentioned OCR errors.


\chapter{Data Preprocessing}
\label{chapter:data preprocessing} 

 
This chapter describes the preprocessing steps applied on the historical news articles.
The garbled OCR text makes data preprocessing mandatory before application of any text mining algorithms. 

\section{Spelling Correction}

Several kinds of spelling errors exist in the data as described in chapter~\ref{chapter:data description}. This chapter first provides a brief review of spelling correction algorithms that exist in literature (Section~\ref{spell:rw}); Section~\ref{spell:algo} describes the algorithm used in this research and evaluation results on the OCR dataset are presented in Section~\ref{spell:eval}.

% DOUBT: Do the following lines need to be removed?
%The garbled OCR dataset  needs to be refined by correcting the text with the help of a human editor manually or automatic spelling corrector. Due to the huge size of dataset, human editing would be extremely time consuming and expensive making it impossible and indicating requirement of a spelling correction technique. 
%The spelling correction of person named entities in the dataset also requires special consideration so that the person named entities with correct spellings get detected as a result which is the main aim of this research.


\subsection{Related Work}
\label{spell:rw}

Kukich\cite{kukich1992techniques} comprehensively discusses various spelling correction techniques based on Non word, Isolated word and Real word spelling errors. N-gram analysis, Dictionary lookup and Probabilistic techniques are used for correcting isolated and nonword errors while Context-Dependent techniques are used mostly for correcting real word errors including the correction of word split and join errors \cite{elmi1998spelling}.

 N-gram techniques work by examining each n-gram in the text string and comparing against a pre-compiled table of n-gram statistics to retrieve the correct word while Dictionary look up techniques directly check whether the text string appears in the dictionary using string matching algorithms. Both techniques require a dictionary or a large text corpus and take frequency of n-grams or word occurrence into account in order to find the correct spelling \cite{strohmaier2003lexical}, \cite{ringlstetter2007text}.
 Probabilistic techniques use transition and word confusion probabilities to estimate likelihood of the correction in order to rank and retrieve correct word spelling.

On the other hand, Context-dependent techniques require contextual information and use either extensive NLP techniques or Statistical Language Modeling (SLM) for spelling correction.
Bassil and Alwani\cite{bassil2012ocr} use Google 1-5 gram word dataset to gain context information in order to determine the correct words sequence in the text for correction.
Tong and Evans\cite{tong1996statistical} use SLM approach involving information from letter n-grams,character confusion and word bi-gram probabilities to perform context sensitive spelling correction obtaining a 60 percent error reduction rate. 
 %In Collection OCR, Sankar et. al. [Sankar K. et al. 2010] use an approximate fast nearest neighbor algorithm based on hierarchical K-Means (HKM) to clean OCR text.

All these spelling correction techniques have developed over time and have been used in combination to achieve improved accuracy \cite{brill2000improved}. \cite{agarwal2013utilizing} use a combination of Google suggestions, LCS and character confusion probabilities for choosing the correct spelling on a small set of historical newspaper data and achieve recall and precision of $51\%$ and $100\%$ respectively.


Edit distance approach, suggested initially by Wagner and Fischer\cite{wagner1974string}, is a dictionary lookup approach commonly used for OCR data correction because of the large number of substitution errors in OCR data  \cite{kukich1992techniques}\cite{christen2006comparison} which can be corrected using this technique. String edit distance approaches with faster correction are discussed in \cite{marzal1993computation},\cite{schulz2002fast}  with variants like Levenshtein automata and normalized edit distance.

Personal name spelling correction has also been studied separately including comparison among various techniques indicating that there is no one single technique that outperforms all others though pattern matching techniques result in better matching quality compared to phonetic encoding techniques\cite{christen2006comparison}. Almost all the personal name spelling correction techniques use personal names directory/dictionary for matching against wrongly spelled names in the dataset or queries in case of People Search \cite{udupa2010hashing}. 
  
%CHANGES MADE HERE. ADDED A LINE REGARDING PERSONAL NAMES CORRECTION AND MOVED RELATED WORK REGARDING EVALUATION ALGORITHM TO SECTION 4.1.3.
 We use edit distance algorithm for spelling correction because of its speed and ability to correct OCR errors compared to the n-gram approach \cite{chattopadhyaya2013fast}. Context-dependent spelling correction is not used because of unavailability of n-gram words corpus or ground truth dataset containing OCR and true word pairs. Our edit distance algorithm also uses an enhanced dictionary for look up to give significance to personal names spelling correction in the dataset. 


\subsection{Spelling Correction Algorithm}
\label{spell:algo}

The Edit Distance algorithm based on Levenshtein distance\cite{levenshtein1966binary} has been used for spelling correction. It is an isolated word correction technique that uses dictionary based-look up method and distance between strings for matching the text and correcting it. An ``edit distance" corresponds to the minimum number of insertions, deletions, and substitutions required to transform one string into another. The algorithm corrects Non-Real Word spelling errors up to an edit distance of 2 , i.e. , it corrects words which have spelling errors that can be corrected by making at most 2 operations of insertion, deletion and substitution of letters in the word. The choice of 2 is governed by the trade off between algorithm runtime and quality of spelling correction. A bigger value improves the spelling correction accuracy but increases the runtime also while a smaller value decreases accuracy and the algorithm runtime.
The spelling corrector has been designed as suggested by Peter Norvig \footnote{ http://norvig.com/spell-correct.html}. The algorithm requires a dictionary which is used to check if each word of the text exists in it or not. If the word already exists in the dictionary then no change is made to the word and if not, then a candidate list of words is created from the word to be corrected by inserting, substituting or deleting up to 2 letters from it.  This list of words is again checked for in the dictionary and returned as suggestions for the word to be corrected. The correction is made with the word formed from lowest edit distance and occurring with more frequency in the dictionary. This makes the edit distance algorithm dependent on the type of dictionary chosen for correction which means the dictionary must be well chosen for spelling correction of a specific document collection. The algorithm runs faster by reading the dictionary only once and keeping a data structure in memory for its word counts which can be referred to whenever a word comes up for correction.

\subsection{Spelling Correction Algorithm Evaluation}


There has not been much related work regarding automatic evaluation of word-by-word post spelling correction on OCR dataset consisting of Word Split and Join errors. Semi-automatic spelling correction system is discussed in \cite{taghva2001ocrspell} that corrects these errors but requires user interaction in order to perform complete correction and system evaluation. Rice\cite{rice1996measuring} discusses OCR errors similar to the ones in our dataset. Their algorithm evaluates edit distance spelling correction by estimating word accuracy defined as the percentage of correctly recognized words; the length of LCS between correct and incorrect strings on a page by page level is used as the relevant metric. The evaluation strategy works correctly but the definition of accuracy does not give a complete coverage of the spell correction as it does not provide any information on the errors missed by the spelling corrector.

For evaluation on our dataset, the raw OCR text and OCR text after application of spelling correction algorithm needs to be compared with the original newspaper text. The OCR text is extremely garbled with Word Split and Join errors due to which word-to-word alignment with the original newspaper text is impossible. For this purpose, a novel algorithm called SCE (Spelling Correction Evaluation) based on N-gram approach is proposed for automatic evaluation of the corrected text word-by-word against the manually corrected subset of the news articles dataset. 

Following sections describe the evaluation parameters for estimating the performance of Spelling Corrector on the OCR dataset used along with the SCE algorithm:

\subsubsection{Evaluation Parameters}

\begin{description}

 \item[$\bullet$Accuracy]
 The evaluation metric used for measuring the performance of Spelling correction algorithm is Accuracy which requires calculation of number of OCR errors that got corrected when compared to the original scanned newspaper text. The measure has been chosen so as to include the complete text coverage and not just check for words that got corrected after spell correction as in the latter case, the  number of FP and TN get missed which won't give the correct measure of how well the spell corrector works. The formula used for calculating Accuracy is defined by Manning and Schutze,1999 (p.268-269) as follows:

$Accuracy=  \dfrac{TP+FP} {TP+ FP + TN + FN}$


where 

TP=Number of True Positives,

TN=Number of True Negatives,

 FP=Number of False Positives,

 FN=Number of False Negatives. 

The aim of the SCE algorithm is to make a word-to-word correspondence between the OCR corrected text and the original OCR text and to mark each token in the OCR text as a TP, FP, TN or FN. Reynaert and Martin\cite{reynaert2008all} suggest a way to define these terms by distinguishing between correct words and incorrect words in the text through the set of non-target, target and selected words and use Precision and Recall evaluation measures for measuring performance of spelling correction. 

According to our spelling corrector, a ``true positive" is said to occur when a word from the OCR text gets corrected and the corrected spelling matches the one in original article text while a ``false positive" occurs if the corrected spelling does not match the corresponding word in the original article text. A ``true negative" occurs when a word does not get corrected by the algorithm as it is already correct and matches the correct word in the original text also. On the other hand, a ``false negative" occurs when the algorithm is unable to correct the word (there is no change in spelling of the word) and it does not match the corresponding word in the original text but should have been corrected.



\item[$\bullet$Time taken for Spelling Correction]
Time is also an essential parameter while measuring the performance of spelling correction. Since the dataset is quite large, it is important that the algorithm does not take too long to correct an article and needs to be parallelized in case it takes more time for correction.


\item[$\bullet$Person Names Detection Rate]

The spelling correction algorithm is evaluated on the basis of another parameter which is used to consider the special case of person entity names spelling correction, as the main goal of research is to detect these names with correct spellings.
Person Names Detection Rate can be defined as the ratio of person names recognized through Named Entity Recognition before spelling correction process and the total number of person names recognized in the original newspaper articles.

$Person Names Detection Rate=\dfrac{ \text{Person Names recognized before/after spelling correction}} {\text{ Person Names recognized in original newspaper articles}} $

\end{description}

\subsubsection{SCE (Spelling Correction Evaluation) Algorithm}

The SCE algorithm is based on an N-word grams approach. To make the correspondence between corrected and original OCR text, a window of n-word grams in the scanned image text article is considered (Original.txt) which can be seen in a diagrammatic representation in Figure 4.1.

\begin{figure} [!htb]
\centering
\includegraphics[scale=0.8]{ngram}
\caption{Schematic diagram for alignment of spell corrected article text with original article text for a word $W_{k}$}
\end{figure}
For each token in the spell corrected text (Corrected.txt), the corresponding token  in the scanned text article along with 2 tokens before and 2 tokens after it are considered for alignment\footnote{ The choice of 2 is based on the Word Split and Join errors in the dataset. Choosing n=2 allows a window of words to be considered to make up for the alignment lost because of Word Split and Join errors.}.
If the token being considered matches with any of these words in the scanned text article words window and its spelling has been corrected when compared to the corresponding token in raw OCR text (OCR.txt), then it is marked as a ``True Positive" which is actually rewarding the Spell corrector for making the correct spelling change. A ``False Positive" is marked if it does not match any of the words despite its spelling being corrected. If the token being considered matches any of the words in the words window but no spelling correction has been made for it, then it is marked as a ``True Negative" and if it does not match any word in the window and the spelling corrector also did not correct it, then it is marked as a ``False Negative" as the word got missed by the corrector. 

Several cases could occur like difference in the lengths of linetext between OCR and Original text or while considering the first, second or the last tokens from the Corrected text for which the corresponding word window in Original text needs to be smaller. All such cases have been covered in SCE Algorithm 1 which calls function `MatchWordGrams' (Algorithm2) for these different cases. 

A limitation of the SCE algorithm is that it requires all 3 versions of a newspaper article (Original, Corrected and OCR) to have the same number of lines. In case of difference in the number of lines of text due to some Word Split and Join errors, the words window needs to be extended so as to cover previous and next line texts also for alignment.


\begin{algorithm}[!h]
\caption{SCE Algorithm for Spell Correction}
  \KwIn{Ocr.txt,Corrected.txt,Original.txt}
  \KwOut{Spell Corrector Accuracy }
\SetKwFunction{MatchWordGrams}{MatchWordGrams}%
 $OcrLine$:=a line of text from Ocr.txt\;
 $CorrectedLine$:=a line of text from Corrected.txt\;
 $OriginalLine$:=a line of text from Original.txt\;
 $tp \leftarrow $0  $fp \leftarrow $0 $tn \leftarrow $0 $fn\leftarrow $0\;  
	\For{(int i=0; i $<$ CorrectedLine.length ; i++) }
	{

    \If{(CorrectedLine.length$<$4 $||$ OriginalLine.length$<$4)}
	{		
    	\MatchWordGrams{(OcrLine,CorrectedLine,OriginalLine,0,OriginalLine.length,i)}\; 
	}
    \Else{
   \If {(i==0)}
   {
\MatchWordGrams{(OcrLine,CorrectedLine,OriginalLine,0,3,0)}\;
   }
   \ElseIf{ (i==1)}
   {
\MatchWordGrams{(OcrLine,CorrectedLine,OriginalLine,0,4,1)}\;
   }
	\ElseIf{(i==(CorrectedLine.length-2) $||$ (CorrectedLine.length-1) $||$ (CorrectedLine.length) $||$ (CorrectedLine.length+1))}
	{
\MatchWordGrams{(OcrLine,CorrectedLine,OriginalLine,i-2,OriginalLine.length,i)}\;
	}  
	\ElseIf{(i $>$= CorrectedLine.length+2)}
	{	
\MatchWordGrams{(OcrLine,CorrectedLine,OriginalLine,OriginalLine.length-3,OriginalLine.length,i)}\;	
	}
	\Else
	 {
\MatchWordGrams{(OcrLine,CorrectedLine,OriginalLine,i-2,i+2,i)}\;	
	}	
   }
 }
 	 $Accuracy=(tp+tn)/(tp+tn+fp+fn);$\
\end{algorithm}


\begin{algorithm}[!htb]
\caption{MatchWordGrams Function called by Algorithm 1}
\begin{algorithmic}
\Function {MatchWordGrams}{OcrLine, CorrectedLine, OriginalLine, jstart, jend, i}
  
 \For{(int j=jstart; j$<$jend; j++)}
  {
    \If{ ((CorrectedLine[i].equals(OriginalLine[j]))\&\&(!(OcrLine[i].equals(CorrectedLine[i]))))}
     {
	  $tp=tp+1$\;
	  flag0=false\;
	 \Return $tp$\;
	  }
	\ElseIf{((CorrectedLine[i].equals(OriginalLine[j]))\&\&(OcrLine[i].equals(CorrectedLine[i])))}
	      {
		 $tn=tn+1$\;
		  flag1=false\;
		\Return $tn$\;
	      }
}

	 \If{(!(OcrLine[i].equals(CorrectedLine[i]))\&\&flag0==true)}
	 {
		    $fp=fp+1$\;
		   \Return $fp$\;
            }
	 
	 \ElseIf{((OcrLine[i].equals(CorrectedLine[i])) \&\& flag1==true)}
	 {
		    $fn=fn+1$\;
		   \Return $fn$\;
	 }
\EndFunction
\end{algorithmic}
\end{algorithm}


\clearpage


\begin{figure} [!htb]
\begin{center}
\includegraphics[scale=0.4]{img3}
\includegraphics[scale=0.75]{originalimg3}
\caption{Scanned image of a newspape article (left) along with its original text (right)}
\end{center}
\end{figure}


\begin{figure} [!htb]
\includegraphics[scale=0.75]{ocr3}
\includegraphics[scale=0.75]{corrected3}
\caption{OCR raw text (left) and Spell corrected text (right) of the article}
\end{figure} 


\textbf{An Example}


Working of the SCE algorithm can be demonstrated with the help of the following example:
Consider 3 versions of a scanned image of a newspaper article,  the original text of the scanned image in Figure 4.2 and the raw OCR text and the corrected text (after spell correction) in Figure 4.3. As highlighted in the figures, for line 6 the line texts are:

 OcrLine= \textit{by tltn rejmrt of th cepert aocountauts who}

CorrectedLine= \textit{by than report of the expert accountants who}

OriginalLine= \textit{by the report of the expert accountants who} 

Here, for each token of CorrectedLine, we find its index and call the MatchWordGrams function accordingly. For the first token 'by' at index i=0 in CorrectedLine, we consider the word window to be "by the report" (index j=0 to 2) in OriginalLine by matching iteratively with each token to see if there is a match and also if there has been a spelling correction by comparing with the corresponding token in OcrLine. Here, no change was made to the spelling of 'by' and it matches with a word in words window, so it is marked as a FN. For the second token 'than' at index i=1, we consider the word window to be "by the report of" (index j=0 to 3) for which there is no match in the window but there has been a spelling correction from 'tltn' to 'than', which implies the correction was wrong and the token is marked as a FP. For the third token 'report' at index i=2, we consider the window as "by the report of the" (index j=0 to 4) in Original Line and find that there is a match in the word window and there has been a spelling correction too from 'rejmrt' to 'report' which makes this token a TP. Similarly, rest of the tokens get marked for each line in the Corrected.txt. 

Another example can be considered from Line 10 in Figure 4.2 and Figure 4.3 where the number of tokens is different in CorrectedLine and OriginalLine. In such a case, direct alignment between tokens is not possible because of which the words window becomes useful. Here, when the last token 'Richmond' of CorrectedLine is considered at index i=3, the corresponding words window becomes "Jury now sitting at Richmond" (index j=1 to 5) for which there is a match in the words windows and corresponding spelling has also been changed from 'tilchmond' to 'Richmond' which makes it a TP. Had the word window not been considered, the corresponding token at index j=3 in OriginalLine would have been chosen as 'sitting' which would have resulted in a FP. 
   


\subsection{Spelling Correction Algorithm Evaluation Results}
\label{spell:eval}

\noindent \textbf{Aims: }The aim of our experiments is to answer the following questions:
\begin{itemize}
\item \textbf{Question 1: }How good is the spell corrector? The metrics for evaluation are accuracy and time to correct the text.
 
\item \textbf{Question 2: }How good is the Person Names Detection Rate? The metric for evaluation is PNDR. 

\end{itemize}

\noindent \textbf{Materials: }
The spelling correction algorithm is used to correct all the 14020 OCR raw text articles in the dataset. The dictionary used for look-up is a concatenation of several public domain books from Project Gutenberg and lists of most frequent words from Wiktionary and the British National Corpus\footnote{http://norvig.com/big.txt}. This is augmented with a large people names list which is obtained  by running Stanford NER-CRF parser on subsets of the ClueWeb12 dataset made available in the TREC 2013 Crowdsourcing Track\footnote{http://boston.lti.cs.cmu.edu/clueweb12/TRECcrowdsourcing2013/}. This enhanced dictionary has been used to give special consideration to correction of person names in the dataset.

\noindent \textbf{Methods: }
In order to answer \textbf{Question 1} we do the following: 

3 versions of each newspaper article are required: OCR raw text, spelling corrector corrected text and the original scanned newspaper article text. Since the dataset is quite large (14020) and it is not possible to get original text of each of these newspaper images, a smaller number of articles are chosen to study the results of spelling correction. 50 scanned newspaper images are taken and an online OCR \footnote{www.onlineocr.net} is run on them followed by some manual correction to get the original articles text. Accuracy can then be calculated for all 3 versions of 50 newspaper articles using the SCE algorithm by marking each word in the OCR text article as a TP, FP, TN or FN. 

In order to answer \textbf{Question 2} we do the following: 
%% AAYUSHEE CHECK--DONE
\begin{enumerate}
\item Person Name Detection from raw OCR (\textbf{Baseline: }) The NER is run on the raw garbled OCR text.
\item Person Name Detection from spell corrected text (\textbf{PND+Spell Correction: })The NER is run on the spell corrected (using the edit distance algorithm) OCR text without the people names list in the dictionary.
\item Person Name Detection from spell corrected text with enhanced dictionary (\textbf{PND+Spell Correction+Enhanced Dictionary: }) The NER is run on the spell corrected (using the edit distance algorithm) OCR text with the enhanced people names list in the dictionary.

\end{enumerate}

\noindent \textbf{Results: }
The spelling corrector shows an Accuracy of $72.7 \%$  when corrected text is compared to OCR text and original article text. We believe that the results are less accurate due to the presence of a large number of Non-word, New Line, Word Split and Join errors in the OCR data which can not be corrected by the spelling correction algorithm used for this research.

The spelling corrector takes 9 seconds on an average to correct the newspaper OCR articles. It takes a total of 36 hours to run on 14020 articles.


%% Aayushee correct this section using the acronyms defined above--DONE
But the spelling correction is useful in terms of Person Names Detection Rate \textit{(PNDR)}. Following are the statistics obtained for \textit{PNDR}:

\textit{PNDR} for \textbf{Baseline: }: 72.5\% 

\textit{PNDR} for \textbf{PND+Spell Correction: }: 63.3\% 

\textit{PNDR} for \textbf{PND+Spell Correction+Enhanced Dictionary: }: 82.5\% 

These statistics indicate that spelling correction using an extended dictionary for personal names is useful for detecting person names from the garbled newspaper articles and the results are dependent on the type of dictionary being used for spell correction.
 

\textbf{Discussion: }

\begin{description}
\item[$\bullet$]\noindent
A better result can be obtained by correcting the New Line errors in the articles. This can be done by checking for if the word at last index of a text line or the word at first index of the next text line is a word not present in the dictionary and combining the two and checking again in the dictionary for a valid word. The new word, if present in the dictionary can be replaced by the two words from which it is formed thereby removing the New Line error. 
\item[$\bullet$]\noindent Similar approach can be applied for Word Split and Join errors but would require each word of an article not present in the dictionary to be analyzed along with some window of words before and after it to make a correction. 
\item[$\bullet$]\noindent How choice of a dictionary for the edit distance algorithm affects the results still remains to be verified. We believe using a historical dictionary can perform spelling correction better and improve the accuracy.
\item[$\bullet$] \noindent Other spelling correction algorithms like context dependent spelling correction can also used to correct the dataset and measure accuracy using our SCE algorithm along with other evaluation parameters to compare among multiple algorithms and decide which one suits the dataset better and gives best accuracy. 
\end{description}



 

\chapter{PEOPLE GAZETTEER}
\label{chapter:people gazetteer}

 %Define ppl gazetteer in chapter 1 and why is it required.
People Gazetteer as defined in Chapter 1 consists of tuples of person names along with list of documents in which they occur and their corresponding topics. This chapter describes the 2-step process of construction of the People Gazetteer by
a) Extraction of person names from the news articles dataset using Named Entity Recognition in  Section~\ref{ner} and
b) Assignment of topics to news articles using LDA topic detection in  Section~\ref{topic detection}.
Combined results of the people gazetteer are presented in Section ~\ref{gaz:result}.

\section{PNER}
\label{ner}


\subsection{Definition}
NER (Named Entity Recognition) refers to classification of elements in text into pre-defined categories such as the names of persons, organizations, locations, expressions of times, quantities, monetary values, percentages, etc. 
Person Named Entity Recognition (PNER) can be defined as the process of NER that marks up only person names that occur in the text.

PNER is required in this research so as to extract all person name entities occurring in the complete dataset and identify influential person entities among them through development of the People Gazetteer. 
PNER aids in the development of People Gazetteer by first extracting all person names occurring in the dataset followed by reverse linking of a person with the articles in which he/she occurs.

\subsection{Tool Used}

The Stanford CRF-NER\footnote{http://nlp.stanford.edu/software/CRF-NER.shtml} is used for PNER in this research. It can perform NER for 3 classes: Person, Organization and Location and is based on linear chain CRF (Conditional Random Field) sequence models. It is trained across several corpora and is fairly robust across multiple domains and even better when compared to some other open source NER systems as illustrated in \cite{rodriquez2012comparison}. According to their results, Stanford NER gave overall the best performance across 2 OCR datasets, and was most effective for PNER when compared with 3 other open source NER systems.


\subsection{PNER Results}

\begin{figure}[h]
\includegraphics{NER1}
%\caption{NER on a news article}
\includegraphics{NER2}
\caption{NER on a sample news article (left) and Table showing output of PNER on 14020 articles (right)}
\end{figure} 
%\begin{figure}[h]

%\end{figure} 
NER on a sample news article from the dataset can be seen in Figure 5.1.
 Stanford NER recognizes a person's full name as separate names by default which is rectified by combining these multi-term entities into single person entities. Person names tagged with ``PERSON" category are stored while running NER on the dataset.
Whenever a person name occurs in a document, the person entity's name along with the document name is stored to obtain tuples of person names with their document lists.
The Stanford NER takes 25 minutes to run on the complete news dataset of 14020 articles extracting a total of 38426 person entities. The output obtained can be seen in Figure 5.2 which shows the number of person entities with the corresponding number of documents in which they occur.  

We divide the people entities extracted into following categories so that separate analysis can be done for each category:
\begin{description}
 \item[$\bullet$Marginally Influential]: This category includes all person entities with occurrence in 4 or below news articles.
\item[$\bullet$Medium Influential]: This category includes all person entities with occurrence between 5 and 15 news articles. 
\item[$\bullet$Highly Influential] : This category includes all person entities with occurrence in more than 15 news articles.
\end{description}



\section{Topic Detection}
\label{topic detection}

 Topic models are algorithms for discovering the main topics that occur across a large and otherwise 
unstructured collection of documents and can organize the collection according to the discovered topics.
Here, a topic refers to a set of words which describe what any document is about.
 A topic model examines the set of documents and discovers based on the statistics of the words in each, what the topics might be and what each document's balance of topics is.
Documents are considered as a mixture of topics and each topic a probability distribution over words.
 Topic detection is the process of identifying topics in a document collection using a topic model. A simple example of topic model illustrated by \cite{blei2012probabilistic} can be seen in Figure 5.2.
Topic detection is essential to this research in order to determine the topics of individual news articles that a person entity occurs in so that the person entity can be linked to the documents in which he/she occurs along with their respective topics.

%DOUBT--Whether to mention if: The discovered topics for articles are further used to detect influential persons across multiple articles topics.


\subsection{Topic Detection Model Used}


\begin{figure}[h]
\includegraphics{topicmodel}
\caption{Simple topic modelling approach for a single article\cite{blei2012probabilistic}.}
\end{figure} 

A simple parallel threaded LDA model described in \cite{newman2007distributed}  is used for topic detection in this research.
LDA (Latent Dirichlet Allocation) is a generative probabilistic model in which each document is modeled as a finite mixture over an underlying set of topics and each topic, in turn, is modeled as an infinite mixture over an underlying set of topic probabilities\cite{blei2003latent}. In other words, documents exhibit multiple topics and each topic is a distribution over a fixed vocabulary. 
%The model was developed as an improved version of Probabilistic Latent Semantic Indexing.
Given an input corpus of `D' documents with `K' topics, the LDA learning process consists of calculating `${\Phi}$', 
maximum-likelihood estimate of model parameters. Given this model, we can infer
topic distributions for arbitrary documents.  
However, this simple LDA approach can take several days to run over a large corpora which is why PLDA suits to large datasets such as ours.


The PLDA model uses distributed computation where total dataset is distributed equally among multiple processors. Initialization involves data and parameters distribution to each processor and random assignment of topics so that each processor has its own copy of data, word topic and topic counts. The topic model inferencing then uses simultaneous local Gibbs sampling approach on each processor for a pre-decided number of iterations to reassign topic probabilities, word topic and topic counts. Global update is performed after each pass by using a reduce-scatter operation to get a single set of counts and obtain final topic assignments.
The PLDA model requires user set parameters before inferencing like Number of topics, Number of words in a topic and Dirichlet parameters. 


\subsection{Results}

Parallel LDA model implemented in the Mallet\cite{McCallumMALLET} toolkit is used for topic detection over the complete dataset of 14020 news articles. 
Topic Modeling is done for 50 iterations and two parallel samplers, which each look at one half the corpus and combine
statistics after every iteration. It takes 2 hours to run PLDA on the complete dataset.
 For each news article, a set of topics with their probability distribution score for the article is obtained out of which the topic with highest topic probability score is associated with the article.  
We obtain different results from topic detection with following parameters setting:
\begin{enumerate}
 \item Number of topics(k)=10, Number of words in a topic=10, Alpha=1/k, Beta=0.01
 \item Number of topics(k)=100,Number of words in a topic=10,Alpha=1/k, Beta=0.01
\end{enumerate}
The set of 10 topics obtained through topic detection on the dataset are illustrated in Figure 5.3. The most common topic among the news articles highlighted is highlighted as Topic 0 with the words: ``club, line,street,game,total,won,team,time,half, race".

\begin{figure}[h]
\includegraphics{TOPICWORDS}
\caption{Top 10 words obtained for each topic with its ID through topic detection on dataset. Highlighted Topic 0 is the maximum occurring topic}
\end{figure} 


The different settings of LDA parameters are experimented further in Chapter ~\ref{influential people detection} in order to understand their effect on influential people detection.
\section{People Gazetteer Output }
\label{gaz:result}

Combining the results from both PNER and Topic detection, the people gazetteer is obtained consisting of a list of articles for each person entity with their corresponding article topics.
A snapshot of the people gazetteer can be seen in Figure 5.4 where each person entity is followed by a list consisting of a text Document ID and its corresponding Topic ID. 
\begin{figure}[!h]
\includegraphics{gazetteer}
\caption{Snapshot of People Gazetteer with Person names, Document list and their corresponding Topic ID}
\end{figure} 


 

\chapter{Influential People Detection}\label{chapter:influential people detection}

This chapter describes the process of detection of influential people from the people gazetteer developed in Chapter~\ref{chapter:people gazetteer} and its results with some case studies. Section~\ref{influential:rw} discusses some related work in the field of influential people detection, Section~\ref{influential:measure} the measures used to define an influential person in the newspaper environment followed by their ranking to obtain top influential persons across each person category with results in Section~\ref{influential:results} and conclusion in Section~\ref{influential:conclusion}.

 
\section{Related Work}
\label{influential:rw}

Influential people detection has been mostly done in the field of social networks, marketing and diffusion research.
\cite{kempe2003maximizing} work on choosing the most influential set of nodes  in a social network in order to maximize user influence in the network. They consider spread of influence from an influential node cascading through the network and influencing other neighborhood nodes but we do not consider the case of a network of connected person entities here where influence score of a person entity is influenced by that of its neighboring nodes. We consider each person entity connected with a number of articles instead and measure the influence of those articles to measure the person entity's influence score .

The most relevant work regarding detection of influential people is described in \cite{agarwal2008identifying} where influential bloggers are identified on a blog site. Influence of each blogger is quantified by taking maximum of the influence scores of each blog posted by a blogger. The influence score of each blog is calculated using parameters of importance in a blogsite like number of posts that refer to the blog, number of comments on the blog, number of other posts that the blog refers to and length of the blog. Influential blogger categories are also created based on their temporal patterns of blog posting. 

\cite{lerman2010using} define popularity of a news story in terms of number of reader votes received by it and predict popularity of a news story over time based on voting history and the probability that a user seeing a story at specific position in a list will vote on it. 


\cite{watts2007influentials}
In the above mentioned works, although the problem description matches with our research problem but the parameters defined to measure influence or popularity cannot be used in the newspaper environment. 


\section{Measuring Influence}
\label{influential:measure}

\begin{algorithm}[!htb]
\caption{Procedure to calculate IPI and rank person entities based on it}
\begin{algorithmic}
\Function {CalculateIPI}{}
  


 \KwIn{PersonCategories, PeopleGazetter(Persons,(DocList,TopicList))}
   $NTF \leftarrow $0,  $NDL \leftarrow $0, $NSIM \leftarrow $0, $DI\leftarrow $0, $UniqT\leftarrow $0, $IPI\leftarrow $0\;  
 \For{(int Category :  PersonCategories)}
  {
    \For{(String PersonName : Persons)}
     {
	   \For{(String doc  : DocList)}
	{	
		$NTF=1+\log GetPersonTF(doc)$;
		$NDL=GetDocLength(doc)/GetAvgDocLength()$;
		$ NSIM=GetTopicSimilarArticles(doc,DocList)$;
		$DI=NDL*(NSIM+NTF)$;
		$DIScoreList.add(DI)$;
 	 }
		$Sort(DIScoreList)$;

		$UniqT=GetUniqueTopics(Person)$;

		$IPI=Max(DIScoreList)+UniqT$;

		$IPIScores.put(PersonName,IPI)$;
       }
	$Sort(IPIScores)$;

	$PrintPersonNameandMaxIPI(IPIScores)$;
}
	 
\EndFunction
\end{algorithmic}
\end{algorithm}





 Figure 6.1 describes the procedure of detection of influential people detection. To measure influence in the newspaper environment and to compare and rank people as influential, we define an Influential Person Index associated (IPI )with each person entity in the people gazetteer. To calculate IPI for each person entity, we first define the Document Index (DI) to measure how each document in the person entity's associated list of documents affects the person's influence score.
Following subsections describe the calculation of DI and IPI of a person entity.

\subsection{Document Index (DI) Calculation}

DI refers to the Document Index for each document in a person's document list in the people gazetteer. Following parameters are considered for the calculation of this index:

\begin{description}
\item[$\bullet$Normalized Document Length (NDL)]
Document Length affects the influence score in the sense that a longer news article in which a person entity occurs is deemed to be more important than a shorter one. It is defined as the number of tokens contained in a news article. Document Length is further normalized by dividing it with the average news article length of 14020 articles in the dataset to get Normalized Document Length as follows: 

$Normalized Document Length (NDL)=\dfrac{\text{Document Length}} {\text{Average Document Length}}$


%\item[$\bullet$IDF]
%IDF is used as a parameter in the calculation of DI of a news article to give weight to the person entity's occurrence in the complete %dataset. It can be calculated as the number of news articles in which a person entity occurs in the complete dataset. It is %equivalent to the length of document list in the people gazetteer for each person entity.

\item[$\bullet$Normalized Term Frequency(NTF)]
Term Frequency (TF) accounts for the number of occurrences of a person entity's name in the complete dataset. The TF of the person for which IPI is to be calculated affects the document's influence score as a high number of occurrences of the person entity in the document make it more influential with respect to that person entity. TF is further normalized and calculated as follows:

$Normalized Term Frequency (NTF)=	1	+\log	$(TF of person entity in current article)


\item[$\bullet$Number of similar articles(NSIM)]
This parameter is used in calculation of the DI by finding topic similarity with other articles  in the document list of the person entity for which IPI is to be calculated. 
The set of topics derived from a corpus can be used to answer questions about the similarity of words and 
documents. Two documents are considered similar to the extent that the same topics appear in those documents. So, for a document $d$ whose DI is to be calculated, we consider 

NSIM= Number of articles with the same topic as that of $d$ in the person's list for whom IPI is to be calculated.

This parameter takes into account the effect of a document's score on a person's IPI when there exist several other documents also of the same type in the person's list. 
%The similarity metric used is Cosine Similarity which can be defined as the measure of similarity between two vectors d1 and d2 that measures cosine of the angle between them.
%define cosine similarity formula here:
%Similarity between d1 and d2=cos($\theta$)= $\dfrac{\text{Dot Product of d1 and d2}} {\text{ Length of d1* Length of d2}}$
%OR DEFINE SIMILARITY IN TERMS OF NUMBER OF ARTICLE WITH SAME TOPICS

\end{description}


DI for each document can be defined using the above parameters with the following formula :


			$DI = NDL *	(  NSIM + NTF )$


\subsection{Calculation of Influential Person Index(IPI)}

Once DI is calculated for each document in a person's list, an index is calculated for the person entity in order to measure its influence in the news dataset and calculate its influential score. The ``Influential Person Index" is defined for this purpose and is calculated as follows:
		

$IPI= max DI(d_1, d_2, ...,d_n)+ UniqT$

where , $max DI(d_1, d_2, ...,d_n)$ takes the maximum Document Index $d_i$ from a document in the person entity's list of  $n$ articles 
and $UniqT$ is the number of unique article topics in which person entity occurs. This parameter is used to account for the fact that a single person entity can be talked about multiple news topics in its document list and to include their effect on its influence score.

Ranking is done across each person category of the people gazetteer to obtain top most influential persons. For this, IPI for each person entity across the person categories are sorted in decreasing order to obtain the top most person entities as most influential.
  


\section{Results}
\label{influential:results}

The statistics obtained regarding each person category of people gazetteer are shown in Table~\ref{table:stats}. 
It can be clearly observed from the table that Highly Influential Persons occur in most number of news articles on an average and with highest average term frequency followed by Medium Influential and Marginal Influential Persons. Document Length directly affects the IPI according to the metric designed implying a high IPI for high document length. But this is not always the case as can be observed from the fact that average document length obtained for Marginally Influential People is high in spite of their Average IPI being low indicating that the varying number of similar articles for each person category as well as its Term Frequency share also play an important part in measuring influence.


\begin{table}
\begin{center}
    \begin{tabular}{|p{2cm}|p{2cm}|p{2cm}|p{3.5cm}|p{2cm}|p{2cm}|p{1.5cm}|}
    \hline
    \textbf{Influential Person Category}  &  \textbf{Number of Person Entities}   & \textbf{Average Number of Documents}  &  \textbf{Most Common Topic Words} &  \textbf{Average Document Length}	&  \textbf{Average Term Frequency}	&	\textbf{Average IPI}\\  \hline
Marginal & 38066 & 1.04 & man ho men night back wa room left house told bad door found turned place ran lie front morning & 2119.6 & 1.07 &	4.71	\\ \hline
Medium & 344 & 5.75 & mr court police judge justice case yesterday street district witness jury charge asked attorney trial arrested lawyer told office & 1976.3 & 6.68 & 24.1	 \\ \hline
High & 16 & 22.8 & piano st rooms car york daily chicago city sunday upright parlor furnished broadway hotel av west train brooklyn monthly & 2971.5 & 29.87	&	101.68	 \\	\hline 
  \end{tabular}
  \end{center}
    \caption {Table illustrating average statistics for each Person Category of People Gazetteer }
\label{table:stats}
\end{table}


Due to the unavailability of ground truth consisting of influential people in the newspaper archives from November-December 1894, there is no way to validate our results. 
To broadly evaluate our results, a simple web search query with the person entity's name in the context of 19th century was done for the top 10 influential persons in each person category.
 Among the top 10 influential persons obtained in each of the highly, medium and marginally influential person categories from the results,  7, 6 and 4 respectively were found to be influential and popular in the 19th century across topic categories like theatre,politics,shipping,etc. when searched on the Web. Most of the false positives although influential in other respects but are not  influential person entities which can attributed to the incorrect PNER (Person Named Entity Recognition) on noisy OCR data.
16 person entities out of the total 30 topmost influential persons in all person categories were found to have an existing page in Wikipedia although it disambiguation is difficult because of incomplete names of the entities like ``William II", ``John II",etc.
   
It was also observed that many of the topmost influential persons from our results didn't make it to the Wikipedia list of significant people in the 19th century\footnote{http://en.wikipedia.org/wiki/19th\_century\#Significant\_people} since the list doesn't consist of influential people found in an American newspaper environment only. Several person entities ranked lower in our results can still be found in this Wikipedia list and others relating to 19th century famous American people.

 A detailed report of the influential persons with their IPI for each person category can be seen in the Appendix (Table ~\ref{table:app1} ~\ref{table:app2} ~\ref{table:app3}).


The Top 3  Influential People detected in each category of person entities is illustrated in Table~\ref{table:pcat1}
It has been found that the boundaries of person categories need not be strict as set up earlier. This can be demonstrated by the fact that a medium influential person ``James McClutcheon" occurring in only 15 news articles has a higher IPI compared to that of the highly influential person ``Alexander III" occurring in 21 new articles. This also illustrates the fact that number of articles of occurrence of an entity in newspaper alone cannot be used to define the popularity of a person. The results obtained also reveal the fact that not all persons with higher IPI might be highly influential. False positives are obtained for person entities  such as ``Mr Got" which is not a person entity and for entities such as ``Ann Arbor" and ``Van Cortlandt" which are in fact locations but got recognized as highly influential person entities.
The influential persons of individual person category are further discussed in the following case studies: 
\begin{table}
\begin{center}
    \begin{tabular}{|l|l|p{3.5cm}|p{4cm}|}
    \hline
    \textbf{PERSON NAME}     & \textbf{IPI}  &  \textbf{NUMBER OF ARTICLES OF OCCURRENCE} &  \textbf{PERSON CATEGORY}\\  \hline
      
    Wilson Barrett &  327.41 &	29 &  Highly Influential\\ \hline
    Marie Antoinette &     264.48 &	31 &	Higly Influential\\ \hline  
    Alexander III & 86.25  &	21 &	Highly Influential\\  \hline
	James McCutcheon & 223.28 &	15	&	Medium Influential\\  \hline
	Capt Creeten & 169.77 &	10	&	 Medium Influential \\ \hline 
	Jacob Schaefer & 145.97 & 	10	&	Medium Influential \\ \hline 
 	Marie Jansen & 104.07 & 	10	&	Medium Influential \\ \hline 
	Phil King&63.11	&	3	&	Marginally Influential \\ \hline
 	Mr Got	&	59.6	& 3	&	Marginally Influential \\ \hline
	 Capt Williams & 56.45		&	3	&	Marginally Influential \\ \hline
 	Capt Schmlttberger	&	56.45	&	3	&	Marginally Influential \\ \hline
  \end{tabular}
  \end{center}
    \caption {Table illustrating top 3 influential people with their IPI in each people category}
\label{table:pcat1}
\end{table}



\subsection{Case Studies}


\begin{enumerate}

\item
Highly Influential Category- The top 3 person entities from Table ~\ref{table:pcat1} with highest IPI in this category have been correctly identified as person entities influencing a number of different news topics as well as large number of articles belonging to the same topic. Considering Wilson Barrett as an example, the person entity has highest IPI of 327.41 occurring in 29 articles, 4 unique topic types and 19 articles belonging to the same topic with the words : ``piano st rooms car york daily chicago city sunday upright parlor furnished broadway hotel av west train brooklyn monthly". The person entity when searched over the Web has been found as an English manager, actor and playwright leading to the conclusion of correctly being identified as being influential although there is no gold standard data available to compare its IPI with other person entities and validate if he is in fact the most influential person entity across the dataset.

\item
Medium Influential Category- The topmost influential entities from Table ~\ref{table:pcat1} in this category can be compared with the ones in highly influential category since their IPI are quite high. But the person with highest IPI in this category ``James McCutcheon" has been wrongly identified as a person entity and is in fact a linen store with many advertisements in news articles. It occurs in 15 news articles with 13 of them belonging to the same topic. On the other hand, ``Jacob Schaefer" and ``Marie Jansen" from this category are correctly recognized as medium influential as they belong to the fields of carrom billiards and theatre respectively and have been recognized as influential along with correct topic labels.
 
\item
Marginally Influential Category- Person entities belonging to this category have extremely low occurrence in news articles although the IPI of topmost influential entities in this category are comparable to those in the other 2 categories. Most of the false positives are obtained in this category since it is difficult to identify an influential person that has low occurrences in articles, unique topic articles as well as low term frequency share.  Person name disambiguation is required for person entities of this category as can be seen from the top influential persons of this category from Table ~\ref{table:pcat1} being ``Mr Got", ``Capt Williams'' , ``Capt Schmittberger". The person entities in this category are also most susceptible to the spelling errors which is the reason for their low value of number of occurrences leading to low IPI values.

\end{enumerate}

 
       
\section{Conclusion}
\label{influential:conclusion}

The problem of finding influential people from historical OCR news repository is studied in this research. We made novel contributions to the problem solution by implementing an evaluation algorithm for measuring accuracy of spell correction on dataset, developing a people gazetteer for facilitating the process of influential people detection and finally defining an influential person in the newspaper community to obtain the top influential people across categories of the people gazetteer.

Most of the problems faced during the research surfaced due to the noise in OCR dataset used. We believe that finding influential persons from newspaper archives can provide much more beneficial results by using better spelling correction algorithms with improved accuracy.

We didn't consider the problem of Named Entity Disambiguation into account while finding influential people in newspaper which is a difficult problem in itself since it is hard to disambiguate among persons with similar names occurring in newspaper articles with multiple topics. The problem still requires research with probably stricter measures of calculation of IPI for each person to compare and rank the topmost influential persons.
  

                                                           


%\newpage
%\bibliographystyle{these}
\bibliographystyle{acm}
%\bibliographystyle{elsart-harv}
%\newpage
%\section{References}
%\bibliography{Library}

\bibliography{aayushee}
\chapter*{Appendix}\label{chapter:appendix} 

\begin{table}[h]
\centering
\begin{tabular}{|l|l|}
\hline
\textbf{PERSON ENTITY}    &       \textbf{IPI}    \\ \hline
wilson barrett   & 327.41235 \\ \hline
marie antoinette & 264.48248 \\ \hline
rob roy          & 216.77463 \\ \hline
ann arbor        & 196.8364  \\ \hline
van cortlandt    & 123.29883 \\ \hline
alexander iii    & 86.25286  \\ \hline
john ii          & 58.9015   \\ \hline
william ii       & 58.764606 \\ \hline
john jacob astor & 55.564274 \\ \hline
william i        & 44.84307  \\ \hline
nicholas ii      & 41.00501  \\ \hline
john j           & 40.21202  \\ \hline
john w           & 31.704508 \\ \hline
john thompson    & 29.755087 \\ \hline
henry w          & 28.562605 \\ \hline
sandy hook       & 22.617697 \\ \hline
\end{tabular}
\caption{Table representing all influential person entities with their IPI in the Highly Influential Persons Category}
\label{table:app1}
\end{table}



\begin{table}[h]
\centering
\begin{tabular}{|l|l|}
\hline
\textbf{PERSON ENTITY}      &  \textbf{  IPI}        \\ \hline
james mccutcheon   & 223.28214  \\ \hline
capt creeten       & 169.77635  \\ \hline
jacob schaefer     & 145.97162  \\ \hline
john martin        & 133.13087  \\ \hline
mrs herrmann       & 120.98331  \\ \hline
capt hankey        & 117.98789  \\ \hline
marie jansen       & 104.073456 \\ \hline
lillian russell    & 102.203674 \\ \hline
nat lead           & 101.46077  \\ \hline
jay gould          & 95.45242   \\ \hline
anthony comstock   & 90.308846  \\ \hline
paul kumar         & 87.10851   \\ \hline
buenos ayres       & 83.4436    \\ \hline
messrs macmillan   & 79.58097   \\ \hline
henry w dreyer     & 79.18698   \\ \hline
north orlich       & 78.085144  \\ \hline
julia l wyman      & 75.19533   \\ \hline
schmitt berger     & 74.93656   \\ \hline
mr john            & 69.086815  \\ \hline
troy saratoga      & 69.006676  \\ \hline
lottie collins     & 68.17696   \\ \hline
lucille hill       & 67.04173   \\ \hline
henry a meyer      & 66.74959   \\ \hline
messrs appleton    & 58.549248  \\ \hline
capt allaire       & 57.140232  \\ \hline
queen victoria     & 56.656094  \\ \hline
jacob brothers     & 56.469116  \\ \hline
frederick baker    & 56.0512    \\ \hline
elliott schenck    & 56.02888   \\ \hline
toledo ann arbor   & 55.86186   \\ \hline
jim hooker         & 54.99624   \\ \hline
lawrence kip       & 53.462437  \\ \hline
john wilson        & 52.78464   \\ \hline
george wilkes      & 48.564274  \\ \hline
charles frohman    & 47.075127  \\ \hline
james dunne        & 46.954926  \\ \hline
john maiden        & 46.412354  \\ \hline
william collier    & 45.601     \\ \hline
mr cesar           & 45.317196  \\ \hline
olga nethersole    & 45.317196  \\ \hline
isaac pitman       & 45.170284  \\ \hline
conan doyle        & 44.292152  \\ \hline
monroe salisbury   & 43.495827  \\ \hline
joseph ii          & 41.270996  \\ \hline
john scott         & 41.12521   \\ \hline
cincinnati jackson & 39.672787  \\ \hline
bret harte         & 39.662773  \\ \hline
daniel webster     & 39.64056   \\ \hline
carrie turner      & 27.390652  \\ \hline
frank w angel      & 27.081802  \\ \hline
\end{tabular}
\caption{Table illustrating the top 50 influential person entities in the Medium Influential Persons Category}
\label{table:app2}
\end{table}




\begin{table}[h]
\centering
\begin{tabular}{|l|l|}
\hline
\textbf{PERSON ENTITY }             & \textbf{IPI}       \\ \hline
phil king                  & 63.11352  \\ \hline
mr got                     & 59.606037 \\ \hline
capt williams              & 56.45242  \\ \hline
capt schmlttberger         & 56.45242  \\ \hline
hun upton                  & 52.06511  \\ \hline
capt pinckney              & 45.98572  \\ \hline
principal mcallister       & 45.954926 \\ \hline
aaron trow                 & 44.96826  \\ \hline
hallen hart                & 44.317196 \\ \hline
louis philippe             & 42.894478 \\ \hline
caleb morton               & 41.244476 \\ \hline
clayton cape vincent       & 41.101837 \\ \hline
north amir                 & 40.430717 \\ \hline
marie clavero              & 40.175583 \\ \hline
john macdonald             & 39.553802 \\ \hline
chauncey scott             & 39.348915 \\ \hline
george iv                  & 38.930504 \\ \hline
capt martens               & 37.96828  \\ \hline
charley grant              & 37.96828  \\ \hline
stewart smith              & 37.96828  \\ \hline
fred hawkins               & 36.742905 \\ \hline
napoleon bonaparte         & 36.597664 \\ \hline
john howard                & 35.709515 \\ \hline
frank i                    & 35.65943  \\ \hline
de korn                    & 35.597664 \\ \hline
rufus jr                   & 35.129013 \\ \hline
john ould                  & 35.043407 \\ \hline
mrs darner                 & 34.498833 \\ \hline
arthur brewer              & 34.370617 \\ \hline
john bunyan                & 33.39399  \\ \hline
olney colne                & 32.821884 \\ \hline
mrs talboys                & 32.52888  \\ \hline
mr left                    & 32.48414  \\ \hline
george green               & 32.477463 \\ \hline
hugh allan                 & 31.599672 \\ \hline
mr mon                     & 31.48414  \\ \hline
hun john u kernan          & 31.138565 \\ \hline
lia peat                   & 31.138565 \\ \hline
vassil iii                 & 31.138565 \\ \hline
hon nicholas k worthington & 31.138565 \\ \hline
harland inrlslve           & 31.138565 \\ \hline
ian hummer                 & 31.138565 \\ \hline
julie m                    & 31.138565 \\ \hline
ian krancltcn              & 31.138565 \\ \hline
madame ilaucho             & 30.99971  \\ \hline
hill junior                & 30.96995  \\ \hline
stockton crescent          & 30.96995  \\ \hline
tex pac                    & 30.604342 \\ \hline
henry wolf                 & 30.469114 \\ \hline
richard v hargett          & 30.343906 \\ \hline
\end{tabular}
\caption{Table showing the 50 topmost influential person entities in the Marginally Influential People Category}
\label{table:app3}
\end{table}



\end{document}
